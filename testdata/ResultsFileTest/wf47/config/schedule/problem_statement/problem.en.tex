\problemname{Schedule}

The Institute for Creative Product Combinations (ICPC) tries to find unusual and
innovative ways to unite seemingly unrelated products or technologies, opening
up new markets and creating new jobs. (For instance, their most recent success
was the ``hairbachi,'' a hair-dryer with a hibachi grill top attachment for
preparing on-the-go hot meals.) The company employs $n$ teams of size $2$ to
research individual products, then members of the different teams get together
to explore ways of combining products.

During the pandemic, the ICPC management organized everyone's schedule in such
a way that there were never more than $n$ people in the office at the same
time, and things ran so smoothly that they continued the process once things
began to return to normal. Here is the scheme they used. Label the teams with
integers $1$ through $n$ and the two people on the $i^{\text{th}}$ team as $(i,1)$ and
$(i,2)$ for each $i$ from $1$ to $n$.  Each week, exactly one person from each
team is allowed in the office, while the other has to stay away.  The employees
$(i,1)$ and $(i,2)$ know each other well and collaborate productively
regardless of being isolated from each other, so members of the same team do not
need to meet in person in the office. However, isolation between members from
different teams is still a concern.

Each pair of teams $i$ and $j$ for $i \neq j$ has to collaborate occasionally.
For a given number $w$
of weeks and for fixed team members $(i,a)$ and $(j,b)$, let $w_1
< w_2 < \ldots < w_k$ be the weeks in which these two team members meet in the
office. The isolation of those two people is the maximum of
$$
\{w_1, w_2-w_1, w_3-w_2, \ldots, w_k - w_{k-1}, w+1-w_k\},
$$
or infinity if those two people never meet.  The isolation of
the whole company is the maximum isolation across all choices of $i$, $j$, $a$,
and $b$.

You have been tasked to find a weekly schedule that
minimizes the isolation of the whole company over a given number $w$ of weeks.

\section*{Input}
The input consists of a single line containing two integers $n$ ($2 \leq n \leq 10^4$) and $w$
($1 \leq w \leq 52$), where $n$ is the number of teams and $w$ is the number
of weeks that need to be scheduled.

\section*{Output}
Output a line containing either an integer representing the minimum isolation
achievable for $n$ teams or the word \texttt{infinity} if no schedule
guarantees that every pair of individuals on different teams can meet. If the isolation is finite,
it is followed by $w$ lines representing a schedule that achieves this
isolation. The $j^{\text{th}}$  line of the schedule is a string of length $n$
containing only the symbols \texttt{1} and \texttt{2}, where the $i^{\text{th}}$  symbol
indicates which of the two members from team $i$ comes into the office on week
$j$.
