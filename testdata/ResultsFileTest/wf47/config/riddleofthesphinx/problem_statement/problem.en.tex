\problemname{Riddle of the Sphinx}

\illustration{0.37}{moreau}{Oedipus and the Sphinx by Gustave Moreau, 1864, public domain}%
%
One of the most remarkable landmarks in Egypt is the Great Sphinx of
Giza, a statue depicting a mythical creature with the head of a human,
the body of a lion, and the wings of an eagle.  Sphinxes were regarded
as guardians in Egyptian and Greek mythologies.  Probably the most
famous sphinx is the one who guarded the Greek city of
Thebes. According to myths, when Oedipus tried to enter the city, the
sphinx gave him the following riddle: ``Which creature has one voice,
but has four feet in the morning, two feet in the afternoon, and three
feet at night?''  As you might have heard, Oedipus correctly answered,
``Man --- who crawls on all fours as a baby, then walks on two feet as
an adult, and then uses a walking stick in old age.''

In this problem, you meet a different sphinx who gives you a somewhat
reversed riddle: ``How many legs do an axex, a basilisk, and a centaur
have?''  While you recognize these as creatures from Egyptian and
Greek mythology, you have no clue how many legs each has (except that
it is a nonnegative integer).  The sphinx sternly instructs you to
not touch anything so you are unable to search for the answer on your
phone.

However, the sphinx allows you to ask her five questions.  In each
question you can ask the sphinx how many legs some number of these
creatures have in total.  For instance, you could ask, ``How many legs
do three basilisks and one axex have in total?'' or ``How many legs
do five centaurs have?''  Seems easy enough, you think, but then you
remember that sphinxes are tricky creatures: one of the sphinx's five
answers might be an outright lie, and you do not know which one.

Write a program to talk to the sphinx, ask the five questions, and
solve the riddle.

\begin{Interaction}
  There are exactly five rounds of questions.  In each question
  round, you must first write a line containing three space-separated
  integers $a$, $b$, and $c$ ($0 \le a, b, c \le 10$), representing
  the question ``How many legs do $a$ axex, $b$ basilisks, and $c$
  centaurs have in total?''  After the question is asked, an input line
  containing a single integer $r$ ($0 \le r \le 10^5$) is available on
  standard input, giving the sphinx's answer to your question.

  After the five rounds of questions, output a line containing three
  space-separated nonnegative integers $\ell_a$, $\ell_b$, and $\ell_c$,
  indicating the number of legs of an axex, a basilisk, and a centaur,
  respectively.
\end{Interaction}

\newpage
