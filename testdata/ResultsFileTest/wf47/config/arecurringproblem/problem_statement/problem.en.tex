\problemname{A Recurring Problem}

You have a very big problem! You love recurrence relations, perhaps
a bit too much. In particular, you are a fan of positive linear
recurrence relations (PLRR), which can be defined as follows. First,
you choose the order $k$ of the relation. Then you choose coefficients
$c_1, c_2, \dots, c_k$, and the first $k$ elements of a sequence $a_1,
a_2, \dots, a_k$. The relation is called ``positive'' if all of these
numbers are positive integers. The rest of the sequence can then be
generated indefinitely using the formula
$$
a_{i+k} = c_1 \cdot a_i + c_2 \cdot a_{i+1} + \dots + c_k \cdot a_{i+k-1} \quad\text{for}\quad i \geq 1.
$$
The Fibonacci sequence is the most famous recurrence of this form, but
there are many others.

In fact, yesterday, in a fit of mad mathematical inspiration, you
wrote down {\em all} possible ways of choosing a positive linear
recurrence relation, and each associated infinite sequence, on some
index cards, one per card. (You have a lot of index cards; you buy in
bulk.) It has all been a bit of a blur. But when you woke up today, you
realized that you do not have a good way to order or count the
PLRRs. You tried just sorting the sequences lexicographically, but
there are too many that start with ``$1$'' --- you will never make it to
the later ones.

Fortunately, inspiration struck again! You realized that you can
instead order the PLRRs lexicographically by the generated part of the
sequence only (that is, the part of the sequence starting after the
initial $k$ values). Ties are broken by lexicographic order of the
coefficients.  For example $k=1$, $c_1=2$, $a_1=2$ comes before $k=2$,
$(c_1,c_2)=(2,1)$, $(a_1,a_2)=(1,2)$, even though the continuation of
the sequence is the same for both. This allows you to properly index
your cards, starting from $1$, with every card being assigned a
number.

Given the number on a card, describe the sequence on it!

\section*{Input}

The input consists of a single line with an integer $n$ ($1 \leq n
\leq 10^9$), the index of the desired PLRR.

\section*{Output}

Output four lines detailing the desired recurrence relation. The first
line contains its order $k$. The second line contains the
$k$ coefficients $c_1, \dots, c_k$. The third line contains the
$k$ starting values $a_1, \dots, a_k$. The fourth line contains
the first $10$ of the generated values.
